\tasknumber{2018}{14} При каких $a$ уравнение $(x-3)(2x-a)=x-3$ имеет ровно один корень?
\Solution Раскроем скобки и перенесем всё в левую часть: \begin{center}
$2x^2-ax-6x+3a=x-3.
2x^2-x(7+a)+3a+3=0.$\end{center}
Т.к уравнение всегда квадратное (старший коэффициент не равен нулю) то, чтобы уравнение имело ровно один корень, необходимо и достаточно, чтобы дискриминант был равен $0$, то есть:
\begin{center}
$(7+a)^2-8(3a+3)=0. \Leftrightarrow 49+14a+a^2-24a-24=0.\Leftrightarrow a^2-10a+25=0 \Leftrightarrow (a-5)^2=0 \Leftrightarrow a=5.$\\
\end{center}
\Answer{$a=5$}