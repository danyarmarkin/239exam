import settings; 
outformat = "pdf";
render = 4;

import unicode;
texpreamble("\usepackage{xcolor}\usepackage{amsmath}\usepackage[russian]{babel}");
defaultpen(font("T2A","cmr","m","n"));

import markers;
import lib_macros;

unitsize(1.2cm);

picture temp_pic;

//Аверьянов 8, 2.3.9.

//=============================
//начинаем рисовать картинку для условия
pair A = (0,2);
pair H = (7,2);
pair K = rotate(63,  A)*H;
pair B = A+unit(H-A)*5;
pair D = A+unit(K-A)*6;
pair E = rotate(117,  B)*A;
pair C = A+unit(E-B)*6;

draw(A--C--B--D--A);



dot("$A$", A, dir(220));
dot("$B$", C, dir(0));
dot("$C$", B, dir(0));
dot("$D$", D, dir(0));





//оформление Дано/Найти
label("\begin{tabular}{l}
%\textbf{Аверьянов 8, №2.3.9.}\\
$ABCD$ -- параллелограмм\\
$\angle B-\angle A = 54^{\circ}$,\\
(?): наибольший угол
\end{tabular}", (7,5));

shipout(prefix="task_pic", format="pdf"); //сохраняем картинку условия в task_pic.pdf
