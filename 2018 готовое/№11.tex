\tasknumber{2018}{11} Конференция длится три дня. В первый и второй день выступают по 15 докладчиков, в третий день -- 20. Какова вероятность того, что доклад профессора М. выпадет на третий день, если порядок докладов определяется жеребьевкой?

\Solution Всего выступают $15+15+20=50$ докладчиков. Тогда вероятность, что профессор М. выступит в третий день равна $P=\dfrac{20}{50}=0,4$\\
\Answer{0,4}