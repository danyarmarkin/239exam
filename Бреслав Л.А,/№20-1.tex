20-1. Найдите большую высоту треугольника со сторонами $3\sqrt{3},$ $\sqrt{11}$ и $4$.

Решение:

Найдем площадь треугольника по формуле Герона:
\begin{multline*}
S = \sqrt{p(p-a)(p-b)(p-c)} =\\
=\sqrt{\dfrac{3\sqrt{3}+\sqrt{11}+4}{2}\cdot \dfrac{3\sqrt{3}-\sqrt{11}+4}{2}\cdot \dfrac{3\sqrt{3}+\sqrt{11}-4}{2}\cdot \dfrac{-3\sqrt{3}+\sqrt{11}+4}{2}}=\\
=\sqrt{\dfrac{(3\sqrt{3}+4)^2-(\sqrt{11})^2}{4}\cdot \dfrac{(\sqrt{11})^2-(4-3\sqrt{3})^2}{4}}=\\
=\sqrt{\dfrac{43+24\sqrt{3}-11}{4}\cdot \dfrac{11-43+24\sqrt{3}}{4}} =\sqrt{\dfrac{24\sqrt{3}+32}{4}\cdot \dfrac{24\sqrt{3}-32}{4}} =\\= \sqrt{\dfrac{1728-1024}{16}} = \dfrac{8\sqrt{11}}{4}.
\end{multline*}

Большая высота проведена к меньшей стороне, то есть к сторое $3\sqrt{3}$. Найдем эту высоту из соотношения $h=\dfrac{2S}{a} = \dfrac{2\cdot\frac{8\sqrt{11}}{4}}{3\sqrt{3}} = \dfrac{4\sqrt{11}}{3\sqrt{3}} = \dfrac{4\sqrt{33}}{27}.$


