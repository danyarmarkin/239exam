
2019-7-1.

Из Петербурга в Москву одновременно отправились курьер и гонец. Курьер проехал с постоянной скоростью весь путь. Гонец проехал первую половину пути со скоростью $102$ км/ч, а вторую половину пути – со
скоростью, на $17$ км/ч меньшей скорости курьера, в результате чего прибыл в Москву одновременно с курьером.
Известно, что скорость курьера меньше $60$ км/ч. Найдите скорость курьера.

Решение.

Обозначим скорость курьера за $v$, а расстояние от Петербурга до Москвы за $S$. Используя условия перечисленные в задачи, составим и заполним таблицу:

\center
\renewcommand\arraystretch{3}
\begin{tabular}{|c|c|c|c|}
\hline
&Скорость& Время & Расстояние\\\hline
Курьер&$v$& $S/v$ & $S$\\\hline
Гонец (первая половина пути)&$102$& $\dfrac{S/2}{102}$ & $S/2$\\\hline
Гонец (вторая половина пути)&$v-17$& $\dfrac{S/2}{v-17}$ & $S/2$\\\hline
\end{tabular}
\flushleft
\renewcommand\arraystretch{1.5}

Поскольку курьер и гонец прибыли в Москву одновременно, то можно составить уравнение, приравняв общее время движения курьера к общему времени движения гонца:
$$\dfrac{S/2}{102}+\dfrac{S/2}{v-17}=\dfrac{S}{v}.$$
Поскольку $S\neq 0,$ то можно обе части этого уравнения сократить на $S$ и домножить на $2$, после чего в уравнении останется только одна неизвестная:
$$\dfrac{1}{102}+\dfrac{1}{v-17}=\dfrac{2}{v}
\Leftrightarrow \dfrac{1}{102}=\dfrac{2}{v}-\dfrac{1}{v-17} \Leftrightarrow \dfrac{1}{102}=\dfrac{v-34}{v^2-17v}\Leftrightarrow$$
$$\Leftrightarrow v^2-17v = 102v-3468 \Leftrightarrow v^2-119v +3468=0 \Leftrightarrow\left[\begin{array}{ll}v= 68,\\v=51.\end{array}\right.$$
И поскольку в условии сказано, что скорость курьера должна быть больше $60$ км/ч, то итоговый ответ $68$ км/ч.

