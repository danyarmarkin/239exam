12-2. Найдите большую высоту треугольника со сторонами $4\sqrt{3},$ $\sqrt{23}$ и $5$.

Решение:

Найдем площадь треугольника по формуле Герона:
\begin{multline*}
S = \sqrt{p(p-a)(p-b)(p-c)} =\\
=\sqrt{\dfrac{4\sqrt{3}+\sqrt{23}+5}{2}\cdot \dfrac{4\sqrt{3}-\sqrt{23}+5}{2}\cdot \dfrac{4\sqrt{3}+\sqrt{23}-5}{2}\cdot \dfrac{-4\sqrt{3}+\sqrt{23}+5}{2}}=\\
=\sqrt{\dfrac{(4\sqrt{3}+5)^2-(\sqrt{23})^2}{4}\cdot \dfrac{(\sqrt{23})^2-(5-4\sqrt{3})^2}{4}}=\\
=\sqrt{\dfrac{73+40\sqrt{3}-23}{4}\cdot \dfrac{23-73+40\sqrt{3}}{4}} =\sqrt{\dfrac{40\sqrt{3}+50}{4}\cdot \dfrac{40\sqrt{3}-50}{4}} =\\= \sqrt{\dfrac{4800-2500}{16}} = \dfrac{10\sqrt{23}}{4}.
\end{multline*}

Большая высота проведена к меньшей стороне, то есть к сторое $4\sqrt{3}$. Найдем эту высоту из соотношения $h=\dfrac{2S}{a} = \dfrac{2\cdot\frac{10\sqrt{23}}{4}}{4\sqrt{3}} = \dfrac{5\sqrt{23}}{4\sqrt{3}} = \dfrac{5\sqrt{69}}{12}.$



