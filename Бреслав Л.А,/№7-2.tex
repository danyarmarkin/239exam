

2019-7-2.

Из Москвы в Петербург одновременно выехали генерал и чиновник. Генерал проехал с постоянной скоростью весь путь. Чиновник проехал первую половину пути со скоростью, меньшей скорости генерала на 13
км/ч, а вторую половину пути – со скоростью 78 км/ч, в результате чего прибыл в Петербург одновременно с
генералом. Найдите скорость генерала, если известно, что она больше 48 км/ч.

Решение.

Обозначим скорость генерала за $v$, а расстояние от Петербурга до Москвы за $S$. Используя условия перечисленные в задачи, составим и заполним таблицу:

\center
\renewcommand\arraystretch{3}
\begin{tabular}{|c|c|c|c|}
\hline
&Скорость& Время & Расстояние\\\hline
Генерал&$v$& $S/v$ & $S$\\\hline
Чиновник (первая половина пути)&$v-13$& $\dfrac{S/2}{v-13}$ & $S/2$\\\hline
Чиновник (вторая половина пути)&$78$& $\dfrac{S/2}{78}$ & $S/2$\\\hline
\end{tabular}
\flushleft
\renewcommand\arraystretch{1.5}

Поскольку генерал и чиновник прибыли в Москву одновременно, то можно составить уравнение, приравняв общее время движения генерала к общему времени движения чиновника:
$$\dfrac{S/2}{78}+\dfrac{S/2}{v-13}=\dfrac{S}{v}.$$
Поскольку $S\neq 0,$ то можно обе части этого уравнения сократить на $S$ и домножить на $2$, после чего в уравнении останется только одна неизвестная:
$$\dfrac{1}{78}+\dfrac{1}{v-13}=\dfrac{2}{v}
\Leftrightarrow \dfrac{1}{78}=\dfrac{2}{v}-\dfrac{1}{v-13} \Leftrightarrow \dfrac{1}{78}=\dfrac{v-26}{v^2-13v}\Leftrightarrow$$
$$\Leftrightarrow v^2-13v = 78v-2028 \Leftrightarrow v^2-91v + 2028=0 \Leftrightarrow\left[\begin{array}{ll}v= 52,\\v=39.\end{array}\right.$$
И поскольку в условии сказано, что скорость курьера должна быть больше $5$ км/ч, то итоговый ответ $52$ км/ч.

